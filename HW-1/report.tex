\documentclass{article}
\usepackage[utf8]{inputenc}
\usepackage{siunitx}
\usepackage[english]{babel}
\usepackage{graphicx}
%\usepackage{physics}
\usepackage[utf8x]{inputenc}
\usepackage[a4paper,top=2cm,bottom=2cm,left=2cm,right=2cm]{geometry}
\usepackage{subfigure}
\usepackage{circuitikz}
%\si{\volt}
\usepackage       {floatflt,epsfig}
\usepackage{verbatim}
\usepackage{color}
\usepackage{caption}
\usepackage{amsmath}
\usepackage{amsfonts}
\usepackage[font=small,format=hang,labelfont={sf,bf}]{caption}
\usepackage{graphics}
\usepackage{ctable}
\usepackage{caption}
\usepackage{pgf}
\usepackage{tikz}
\usepackage{listings}

\title{\BIG{Advanced Machine Learning} \\
Exercise 1 \\ Image Filtering and Object Identification}
\author{Abbonato Diletta, Pezone Francesco, Testa Lucia}
\date{25 March 2020}

\begin{document}

\maketitle
\newpage
\section{Question 1: Image Filtering }

\subsection{1.d} \label{1.d}
 
We want to study the effect of applying a filter to an image by looking at its impulse response.

First of all we must note that the convolution is both  commutative and associative, therefore we have:

\begin{equation}
\centering
\left\{\begin{array}{l} f * g =g * f \\ \\ (f*g)*h = f*(g*h)\end{array}\right.
\label{numero alberi}
\end{equation}

The other important aspect is that a Gaussian filter $G$ is separable, which means that we can write it as $G = uv^T$ with $u$ a row vector and $v^T$ a column vector.\\
So we apply first a 1d gaussian kernel to a picture and after its transpose version gives the same result of applying a 2d kernel, we can see this from the top left image in Figure \ref{all_filt_gauss}.

\begin{figure}[!h]
    \centering
	\mbox{\epsfig{file=1_d.png, width=15cm}}
	\caption{Results of differents combinations of pairs of 1d filters for a impulsive signal}
	\label{all_filt_gauss}
\end{figure}

We use the separated version of the 2d gaussian kernel since it requires a small number of products: the 2d kernel requires
$k^2 n^2$ operation, if k is the size of the kernel and n is the size of the image.\\
The separable version requires
only $2kn^2$ operations.
Now let's see the photo in the upper right and the one in the center. Both filters are equal for the commutativity of the convolution; here, we are smoothing along one direction (the x axis) and trying to identify the edges along another (the y axis). The same thing, but with the inverted axes, happens in the images below those of before.\\
With the filter at the bottom left we are unable to detect edges parallel to the axes (x and y), we can only identify edges along mixed directions (f(x,y)) or corners (of course a single pixel is a kind of "corner").
\newpage
\subsection{1.e}

Let's start by commenting on the two images on which the directional 
derivatives kernels have been applied.\\
To better understand, let's focus on the image 'gantrycrane.png', since it has 
many vertical and horizontal lines.\\
These lines are highlighted in two different ways by the two filters: 
the contours of the vertical pole in the center of the photo are seen only 
through the filter with the derivative along the x axis and vice versa.

The important part is to note that the image filtered in this way is strongly 
conditioned by the presence of noise. 
To limit this dependence, it is useful to previously filter the image  with 
a Gaussian filter; let's see the difference below.



\begin{figure}[!h]
	\begin{center}
		\subfigure[graph.png filtered along different axes in the first two images (x and y respectively) and subsequently with the Pythagorean theorem to match the derivatives of both directions]{
			\includegraphics[scale=0.8]{1_2_mur.png}
			\label{or_0,65}
		}
		\hspace{2mm}
		\subfigure[gantrycrane.png filtered along different axes in the first two images (x and y respectively) and subsequently with the Pythagorean theorem to match the derivatives of both directions]{
			\includegraphics[scale=0.8]{1_e_air.png}
			\label{or_0,80}
		}
		
	\end{center}
	\captionsetup{justification=raggedright,margin=1cm}
	\caption{Filtered images}
	\label{Filtered_all}
\end{figure}

As we can see, in Figure \ref{all_filt_gauss} the images on the left have a lot of noise (in the form of very
thin net lines) but as soon as we apply a Gaussian filter this noise is 
smoothed and these lines disappear.\\
In general, we always do this: first we filter to reduce noise and
then we filter to detect edges.


\begin{figure}[!h]
	\begin{center}
		\subfigure[graph.pngfiltered directly with the derivative kernel, on the left. On the right we first apply a gaussian filter with $\sigma = 7$. ]{
			\includegraphics[scale=.30]{1_e_Mur.png}
			\label{or_0,65}
		}
		\hspace{2mm}
		\subfigure[gantrycrane.png filtered directly with the derivative kernel, on the left. On the right we first apply a gaussian filter with $\sigma = 7$. ]{
			\includegraphics[scale=.30]{1_e_Airr.png}
			\label{or_0,80}
		}
		
	\end{center}
	\captionsetup{justification=raggedright,margin=1cm}
	\caption{Derivate filter with or without pre gaussian smoothing}
	\label{Filtered_all}
\end{figure}


\section{Question 3: Object Identification }

From the following table, we can study in more detail the ability of each type of histogram and distance in comparing and recognizing images. We can see that in the first entries of the table the type of distance is the intersection, while the type of histogram is RGB. Regardless of the number of bins, which in the case of our analysis were 5,10,20,30,40. Below we can find some of the results of the rg histogram, and we can verify that the best performance is obtained by the intersection distance. The distance $\Chi^2$ also does not exceed a correctness of 61\%, while in $l2$ it does not exceed 57\%. The best results with regards to the number of bins seem to be the intermediate values. We find that the best performance for RGB has 30 bins, for RG 40 and for dxdy 20. Finally, we can see how the results for dxdy are the worst, since we have a maximum of 60\% (dxdy, intersect, 20 bins) at a minimum of 35\% (dxdy, l2, 5 bins)\\

\begin{center}
\begin{tabular}{llrr}
\toprule
hist type &  dist type &  num bins &  num of correct over 100 \\
\midrule
      rgb &  intersect &        30 &                       72 \\
      rgb &  intersect &        20 &                       71 \\
      rgb &  intersect &        40 &                       70 \\
      rgb &  intersect &        10 &                       70 \\
      rgb &  intersect &         5 &                       69 \\
       rg &  intersect &        40 &                       67 \\
       rg &  intersect &        30 &                       65 \\
       rg &  intersect &        20 &                       65 \\
       rg &  intersect &         5 &                       63 \\
       rg &  intersect &        10 &                       62 \\
      rgb &       chi2 &         5 &                       61 \\
      rgb &       chi2 &        10 &                       60 \\
     dxdy &  intersect &        20 &                       60 \\
     dxdy &  intersect &        40 &                       58 \\
       rg &       chi2 &         5 &                       58 \\
     dxdy &  intersect &        30 &                       57 \\
      rgb &         l2 &         5 &                       57 \\
       rg &         l2 &         5 &                       55 \\
      rgb &         l2 &        10 &                       54 \\
     dxdy &  intersect &        10 &                       53 \\
       rg &       chi2 &        10 &                       53 \\
       rg &         l2 &        10 &                       52 \\
       rg &       chi2 &        20 &                       51 \\
      rgb &       chi2 &        20 &                       48 \\
     dxdy &       chi2 &        10 &                       44 \\
     dxdy &  intersect &         5 &                       43 \\
       rg &         l2 &        20 &                       43 \\
      rgb &         l2 &        20 &                       42 \\
       rg &       chi2 &        30 &                       42 \\
     dxdy &       chi2 &        20 &                       41 \\
       rg &       chi2 &        40 &                       40 \\
     dxdy &       chi2 &        30 &                       40 \\
     dxdy &         l2 &        10 &                       40 \\
     dxdy &         l2 &        30 &                       40 \\
       rg &         l2 &        30 &                       39 \\
     dxdy &       chi2 &         5 &                       39 \\
     dxdy &       chi2 &        40 &                       39 \\
     dxdy &         l2 &        20 &                       39 \\
      rgb &       chi2 &        30 &                       38 \\
     dxdy &         l2 &        40 &                       38 \\
       rg &         l2 &        40 &                       37 \\
     dxdy &         l2 &         5 &                       35 \\
      rgb &         l2 &        30 &                       34 \\
      rgb &       chi2 &        40 &                       34 \\
      rgb &         l2 &        40 &                       30 \\
\bottomrule
\end{tabular}
\end{center}

\newpage
\section{Question 4: Performance Evaluation }
\subsection{RG histograms generated with different number of bins}

\begin{figure}[!h]
	\begin{center}
		\subfigure[RG histograms $bin=5$]{
			\includegraphics[scale=0.30]{rg_histograms_5.png}
		
		}
		\hspace{2mm}
		\subfigure[RG histograms $bin=10$]{
			\includegraphics[scale=0.30]{rg_histograms_10.png}
		
		}
		\hspace{2mm}
		\subfigure[RG histograms $bin=50$]{
			\includegraphics[scale=0.30]{rg_histograms_50.png}
		
		}
		
		
	\end{center}
	\captionsetup{justification=raggedright,margin=1cm}
	\caption{RPC for RG histograms with a variable bin number}
	\label{gr}
\end{figure}
\subsection{RGB histograms generated with different number of bins}
\begin{figure}[!h]
	\begin{center}
		\subfigure[RGB histograms $bin=5$]{
			\includegraphics[scale=0.30]{rgb_histograms_5.png}
			\label{b5}
		}
		\hspace{2mm}
		\subfigure[RGB histograms $bin=20$]{
			\includegraphics[scale=0.30]{rgb_histograms _20.png}
			\label{b10}
		}
		\hspace{2mm}
		\subfigure[RGB histograms $bin=50$]{
			\includegraphics[scale=0.30]{rgb hist 50.PNG}
			\label{b20}
		}
		
		
	\end{center}
	\captionsetup{justification=raggedright,margin=1cm}
	\caption{RPC for RGB histograms with a variable bin number}
	\label{rgb}
\end{figure}
\subsection{Dxdy histograms generated with different number of bins}
\begin{figure}[!h]
	\begin{center}
		\subfigure[dxdy histograms $bin=5$]{
			\includegraphics[scale=0.30]{dxdy hist 5.png}
			\label{d5}
		}
		\hspace{2mm}
		\subfigure[dxdy histograms $bin=20$]{
			\includegraphics[scale=0.30]{dxdy hist 20.png}
			\label{d10}
		}
		\hspace{2mm}
		\subfigure[dxdy histograms $bin=50$]{
			\includegraphics[scale=0.30]{dxdy hist 50.png}
			\label{d20}
		}
		
	\end{center}
	\captionsetup{justification=raggedright,margin=1cm}
	\caption{RPC for dxdy histograms with a variable bin number}
	\label{dxdy}
\end{figure}




With reference to the plots obtained by varying the histogram type and at the same time the number of bins per histogram, it is possible to note the consistency of the results according to the output of point 3.c.
In fact, there are better performances with the use of the RGB histogram. Furthermore, by paying attention to the plot, it’s possible to notice that the intersect distance is the best as regards the use of the RGB itself.
In general, we can also say that the distances l2 and chi2 are less performing than the intersect mode, and this could be well understood from the results obtained previously in which the scores for the distance l2 and the distance chi2 are lower.

Finally, the most significant result is the one  concerning edge detection, using the $dxdy\_ histogram$.
There was already a very low rate in the results of 3.c, but the precision recall plots  confirm these results in which there is a trend that starts from the bottom and continues up very slowly. In addition, we can notice that the three distances do not particularly prevail over each other. This is an indication of the poor effectiveness of the type of histogram, regardless of the $dist\_ mode$ used.

 









\end{document}
